\documentclass[a4paper,12pt]{article}

% template by Bernardo Pandolfi Costa

\usepackage{graphicx}
\usepackage[
top=3cm,
bottom=2cm,
left=3cm,
right=2cm,
]{geometry}
\usepackage{pdfpages}
\usepackage{hyperref}
\usepackage{indentfirst}
\usepackage{listings}
\usepackage{xcolor}
\usepackage{textcomp}
\usepackage[portuguese]{babel}
\usepackage[section]{placeins}
\usepackage[utf8]{inputenc}
\usepackage[T1]{fontenc}
\usepackage{fancyhdr}
\usepackage{lipsum}
% Turn on the style
\pagestyle{fancy}
\usepackage{float}
\usepackage{amsmath}
\usepackage{steinmetz}
\fancyfoot{}

\fancyfoot[R]{\thepage}
\setlength{\headheight}{16pt}% ...at least 51.60004pt

\definecolor{codegreen}{rgb}{0.48,0.7,0.42}
\definecolor{codegray}{rgb}{0.5,0.5,0.5}
\definecolor{codepurple}{rgb}{0.5,0,0.82}
\definecolor{codeblue}{rgb}{0.5, 0.64, 0.74}
\definecolor{backcolour}{rgb}{0.95,0.95,0.95}

\lstdefinestyle{codestyle}{
    backgroundcolor=\color{backcolour},   
    commentstyle=\color{codegray},
    keywordstyle=\color{codepurple},
    numberstyle=\tiny\color{codeblue},
    stringstyle=\color{codegreen},
   	breakatwhitespace=false,         
    breaklines=true,                 
    captionpos=b,                    
    keepspaces=true,                 
    numbers=left,                    
    numbersep=5pt,                  
    showspaces=false,                
    showstringspaces=false,
    showtabs=false,                  
    tabsize=2,
    basicstyle=\ttfamily
}
\usepackage[utf8]{inputenc}
\usepackage[T1]{fontenc}

\lstset{
inputencoding=utf8,
upquote=true,
style=codestyle,
    literate=%
    {á}{{\'a}}1
    {à}{{\`a}}1
    {ã}{{\~a}}1
    {â}{{\^a}}1
    {é}{{\'e}}1
    {ê}{{\^e}}1
    {í}{{\'i}}1
    {ó}{{\'o}}1
    {õ}{{\~o}}1
    {ô}{{\^o}}1
    {ú}{{\'u}}1
    {ü}{{\"u}}1
    {ç}{{\c{c}}}1
    {Á}{{\'A}}1
    {À}{{\`A}}1
    {Ã}{{\~A}}1
    {Â}{{\^A}}1
    {É}{{\'E}}1
    {Ê}{{\^E}}1
    {Í}{{\'I}}1
    {Ó}{{\'O}}1
    {Õ}{{\~O}}1
    {Ô}{{\^O}}1
    {Ú}{{\'U}}1
    {Ü}{{\"U}}1
    {Ç}{{\c{C}}}1
}

\usepackage{helvet}
\renewcommand{\familydefault}{\sfdefault}

\hypersetup{
    colorlinks,
    citecolor=black,
    filecolor=black,
    linkcolor=black,
    urlcolor=blue
}

\author{Bernardo Pandolfi Costa}
\title{Lista de Exercícios 1}

\renewcommand{\contentsname}{Sumário}
\renewcommand{\figurename}{Figura}


\begin{document}
\makeatletter
\begin{titlepage}
    \begin{center}
    \includegraphics[width=64px]{./assets/vertical_sigla_fundo_claro.png}\\
        \small
        \vspace{0.25cm}
        \textbf{UNIVERSIDADE FEDERAL DE SANTA CATARINA}\\
		\textbf{CENTRO DE CIÊNCIAS, TECNOLOGIAS E SAÚDE}\\
		\textbf{DEPARTAMENTO DE COMPUTAÇÃO}\\
		\textbf{CURSO DE ENGENHARIA DE COMPUTAÇÃO}\\
        \vspace{7cm}
        \Large
        \textbf{\@title}\\
        \vspace{0.5cm}
        \normalsize
        \textbf{\@author}\\
        \vspace{1.5cm}
        \small
        Disciplina: Sistemas Operacionais\\
        Professor: Roberto Rodrigues Filho
        \vfill
        Araranguá\\
        \today
    \end{center}
\end{titlepage}
\makeatother

\tableofcontents
\newpage

\section{Exercício 1}
O programa pede ao usuário um comando a ser executado e depois o executa.\\
Exemplos: 
\begin{itemize}
\item ls .
\item ls . -la
\item touch . hello
\end{itemize}

\section{Exercício 2}
Para cada processo, o $A$ manteve um valor diferente. Quando $A$ é colocado em cada processo, a variável global é copiada com valor não alterado, neste caso, $A = 1$. Como cada processo é isolado, a alteração feita no processo filho não é carregada ao processo pai, de forma que a variável $A$ terá um valor diferente em cada processo, baseado na alteração de cada um.\\
No código, foi programado que o processo filho executa uma multiplicação de $10$ em $A$, enquanto o processo pai multiplicará $A$ por $-10$, para que a mudança seja mais perceptível. Assim, no processo filho o resultado final de $A$ foi de $10$ e o do processo pai foi de $-10$.

\section{Exercício 3}
Cada \textit{thread} executa ao mesmo tempo, mas seus valores são compartilhados, visto que a variável $A$ é global. Então as alterações que são feitas em cada \textit{thread} executa sua função independentemente das outras, mas os resultados são compartilhados.\\
Neste código, foi programado que a \textit{thread} principal (dentro do main) multiplica $A$ por $10$ e a \textit{thread} secundária multiplicará por $-10$, de forma similar ao exemplo anterior. Para facilitar a visualização, foi adicionado um atraso na multiplicação de $-10$ para que seja perceptível que a multiplicação de $-10$ na \textit{thread} secundária é feita \textbf{após} a multiplicação da \textit{thread} principal, apesar de ocorrerem independentemente.

\section{Exercício 4}
No código, através das bibliotecas \textit{BeautifulSoup} e \textit{requests}, foi possível acessar os conteúdos de uma página na web sem a necessidade de salvá-la na máquina. Assim, o programa se conecta ao \textit{url} pedido, consegue o corpo do HTML e busca a quantidade de aparições da palavra especificada e retorna o valor ao terminal.

\section{Exercício 5}
De forma similar ao exercício anterior, as bibliotecas \textit{BeautifulSoup} e \textit{requests} foram utilizadas. Neste caso, a \textit{String} que contém os conteúdos da página é divida em três partes de tamanhos iguais e, na criação das \textit{threads}, cada uma recebe uma parte da \textit{String}. Assim, cada \textit{thread} buscará a quantidade de vezes que a palavra aparece em sua respectiva seção do HTML, somando as aparições concomitantemente à mesma variável.

\section{Exercício 6}
O código busca o conteúdo do body da página e o coloca em uma grande \textit{String}, chamada de \textit{body\_text} no código.
Em seguida, a variável \textit{body\_text} é divida em três outras de tamanhos iguais, para que cada pedaço seja colocado em uma \textit{thread} específica. Antes das \textit{threads} serem criadas, uma variável que carrega o momento de início da operação \textit{multithread} é instanciada pelo método \textit{time()}. 
Ao fim do processo, outra variável que indica o momento em que a operação chega ao fim é criada. O resultado de tempo $\Delta t$, então, equivale à subtração do tempo final menos o tempo inicial, que é mostrado no terminal.\\
Ao fim do processo \textit{multithread}, a operação é repetida, mas desta vez, com apenas uma \textit{thread}. Os resultados indicam, então, que o programa em \textit{monothread} é de $3$ a $4$ vezes mais rápido que o código em \textit{multithread}.
Existem alguns possíveis motivos para este resultado. Como instanciar e gerenciais \textit{threads} demanda tempo e poder computacional, quando estamos lidando com tarefas simples, como contar as ocorrências de uma palavra, pode ser que a estratégia não compense.\\
Outra explicação ao ocorrido se deve a como códigos em Python são interpretados pelo GIL (Global Interpreter Lock) em CPython. Este interpretador executa o código em apenas um processo e faz com que os \textit{threads} do código não sejam verdadeiramente paralelos como teoricamente deveriam.

\end{document}